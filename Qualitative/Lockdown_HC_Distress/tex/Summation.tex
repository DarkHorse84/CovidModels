\section{Parting Shots}
When I built these models, I was attempting to understand the impact of health care system distress. What I did not expect to discover was the value of lock downs and their ability to reduce death rates. This can be amplified if lock downs are used to increase the health care systems capacity, something not included in these models.
Another simplifying feature in these models is that the health care system is perfectly fine and suddenly is distressed when a certain threshold is breached. This won't be the case in reality. The health care system will become increasingly less effective as the active infection count increases. It also won't suddenly recover once the active infection count reduces below the threshold. The health care system will take time to recover. This will increase the death rate if the health care system is strained.
It is also useful to remember that these model all assumed a total population size of 15 million people. 2\% is 300,000 hypothetical people, 5\% is three quarters of a million. 