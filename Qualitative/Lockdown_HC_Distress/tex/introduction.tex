\section{Introduction}

Only those that have truly been living under rocks haven't heard the terms "flatten the curve", "Lock down" and "health care system distress". We all know what these words mean but we may not have any sort of instinctual of what this means. Here in South Africa, our government has lifted restrictions from level 5 (or hard lock down) to level 4 (or a softer, friendlier hard lock down). When it comes to modeling pandemics (you can get a master class on this online these days), what does this mean for infection rates and more importantly, fatality rates. Here, I'll be examining and comparing a number of qualitative disease models to understand what this means. There is a caveat though. 'Qualitative' in this context means that the models behave in more or less the correct fashion but the actual numbers do not reflect reality. While qualitative models are not useful in any predictive sense, they do help us understand and get a feeling for the world we live in. For those that are interested in more details, the OpenModelica models, Octave scripts, batch scripts and the tex source for this document can be found by following this link: \url{https://github.com/DarkHorse84/CovidModels/tree/master/Qualitative}.